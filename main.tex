% Modelo UNISINOS para teses e dissertacoes

% PARA COMEÇAR A ESCREVER NÃO SE DETENHA AOS CÓDIGOS INICIAIS DO DOCUMENTO, VOCÊ IRÁ ENTENDÊ-LOS COM O TEMPO. 
% vÁ DIRETO PARA O CAPÍTULO 1 - INTRODUÇÃO.

% Arquivo de configurações e pacotes.
\input{0-setup}

% Arquivo de dados do documento: título, autor...
% ||||||||||||||||||||||||||||||||||||||||||||||
% Informações de dados para CAPA e FOLHA DE ROSTO
% ||||||||||||||||||||||||||||||||||||||||||||||
\titulo{Título} % Não utilize o ponto final no título
\autor{Autor}
\local{São Leopoldo}
\data{2016}
\orientador{Prof. Dr. Nome Sobrenome}
\coorientador{Prof. Dr. Nome Sobrenome} % comente esta linha caso nao tenha coorientador
\instituicao{%
  UNIVERSIDADE DO VALE DO RIO DOS SINOS - UNISINOS
  \par
  UNIDADE ACADÊMICA DE PESQUISA E PÓS-GRADUAÇÃO
  \par
  PROGRAMA DE PÓS-GRADUAÇÃO EM ENGENHARIA ELÉTRICA
  \par
  NÍVEL MESTRADO PROFISSIONAL}
\tipotrabalho{Dissertação (Mestrado)}
% O preambulo deve conter o tipo do trabalho, o objetivo, 
% o nome da instituição e a área de concentração 

\preambulo{Dissertação apresentada como requisito parcial para obtenção do título de Mestre em Engenharia Elétrica, pelo Programa de Pós-Graduação em Engenharia Elétrica da Universidade do Vale do Rio dos Sinos - UNISINOS.}

%\preambulo{Trabalho apresentado como requisito para a obtenção do título de Mestre, pelo Programa de Pós-Graduação em Engenharia Elétrica da Universidade do Vale do Rio dos Sinos – UNISINOS.}

% ----------------------------------------------
% Configurações de aparência do PDF final
% ----------------------------------------------
% alterando o aspecto da cor azul
\definecolor{blue}{RGB}{41,5,195}

% alterando o aspecto da cor cinza
\definecolor{gray}{RGB}{50,50,50}

% informações do PDF
\makeatletter
\hypersetup{
     	%pagebackref=true,
		pdftitle={\imprimirtitulo}, 
		pdfauthor={\imprimirautor},
    	pdfsubject={\imprimirpreambulo},
	    pdfcreator={LaTeX - abnTeX2 - Overleaf},
		pdfkeywords={abnt}{latex}{abntex2}{trabalho acadêmico}{unisinos}{ppgee}{mpee}{mestrado profissional},  
		colorlinks=true, % false: boxed links; true: colored links
    	linkcolor=black, % color of internal links
    	citecolor=black, % color of links to bibliography
    	filecolor=blue,  % color of file links
		urlcolor=gray,	 % color of url links
		bookmarksdepth=4
}
\makeatother

% ----------------------------------------------
% Início do documento
% ----------------------------------------------
\begin{document}

% ----------------------------------------------
% Adiciona lista de correções no início do documento.
% Comentar a linha abaixo quando o trabalho for concluído
% ----------------------------------------------
%\listoftodos
% ----------------------------------------------

% Arquivo de elementos pré-textuais: capa, folha de rosto, ficha catalografica, errata, folha de aprovação dedicatória, agradecimentos, epígrafe, resumos, lista de abreviaturas e siglas, lista de símbolos... 
\input{2-pretextual}

% ----------------------------------------------
% ELEMENTOS TEXTUAIS
% ----------------------------------------------
\textual

% ----------------------------------------------
% Exemplo de capítulo sem numeração, mas presente no Sumário
% ----------------------------------------------
%\chapter*[Introdução]{Introdução}
%\addcontentsline{toc}{chapter}{Introdução}
% ----------------------------------------------

% ||||||||||||||||||||||||||||||||||||||||||||||
% CAPITULO 1 - INTRODUÇÃO
% ||||||||||||||||||||||||||||||||||||||||||||||
% Nas seções e subseções a primeira letra de cada palavra devem estar em maiúsculas.
\chapter{Introdução}\label{Introdução}

\todo[inline, color=yellow]{exemplo de comentário para auxiliar na lista de tarefas e correções}

\todo[inline]{Alguns manuais de pacotes latex foram adicionados na pasta manuals}

\todo[inline, color=red]{Procure se informar a respeito do Mendeley. É um software para gerenciar referencias bibliográficas que irá facilitar muito a inclusão das referências no documento. Procure também como importar as referencias inseridas no mendeley para o arquivo references.bib aqui no overleaf}

Este documento e seu código-fonte são exemplos de referência de uso da classe \textsf{abntex2} e do pacote \textsf{abntex2cite}. O documento exemplifica a elaboração de trabalho acadêmico (tese, dissertação e outros do gênero) produzido conforme a ABNT NBR 14724:2011 \emph{Informação e documentação - Trabalhos acadêmicos - Apresentação}.

A expressão ``Modelo Canônico'' é utilizada para indicar que \abnTeX\ não é modelo específico de nenhuma universidade ou instituição, mas que implementa tão somente os requisitos das normas da ABNT. Uma lista completa das normas observadas pelo \abnTeX\ é apresentada em \citeonline{abntex2classe}.

Este documento deve ser utilizado como complemento dos manuais do \abnTeX\ \cite{abntex2classe,abntex2cite,abntex2cite-alf} e da classe \textsf{memoir} \cite{memoir}. 

Esperamos, sinceramente, que o \abnTeX\ aprimore a qualidade do trabalho que você produzirá, de modo que o principal esforço seja concentrado no principal: na contribuição científica.


\begin{align}
  B'&=-\nabla \times E,\\
  E'&=\nabla \times B - 4\pi j,
\end{align}

\begin{equation} 
 f(x) n=(x+a)(x+b)
\end{equation}

\section{Justificativa}

\begin{figure}[htp]
	\centering
	\caption{\label{fig:met-disc-fig01} No centro da figura é representado um logo $U_1$.} 
	\includegraphics[width = 0.8\linewidth]{images/unisinos.png}
	\legend{Fonte: Desenvolvido pelo autor.}
\end{figure}

\section{Delimitação do trabalho}

\section{Objetivos}

Matematicamente, o Fator de Potência (FP) pode ser expresso como:
\begin{equation}
	\label{eq:k-55}
    {
    \displaystyle 
    FP = \frac{\cos(\varphi)}{\sqrt{1 - THD^2}}
    }
\end{equation}

\subsection{Objetivos gerais}

\subsection{Objetivos específicos}

%%%%%%%%%%%%%%%%%%%%%%%%%%%%%%%%%%%%%%%%%%%%%%%%%
% Capitulo de exemplos utilizando arquivo externo 
%%%%%%%%%%%%%%%%%%%%%%%%%%%%%%%%%%%%%%%%%%%%%%%%%
\include{abntex-exemplos} % comente esta linha para facilitar, mas não apague o arquivo abntex-exemplos.tex pois ele contém exemplos interessantes que podem auxiliar na elaboração da dissertação.

% ||||||||||||||||||||||||||||||||||||||||||||||
% CAPITULO 2
% ||||||||||||||||||||||||||||||||||||||||||||||
\chapter{Revisão Bibliográfica}

\section{Aliquam}

\lipsum[2-3]

\begin{figure}[htp]
	\centering
	\caption{\label{fig:inrush-fig02} Logo Unisinos.}
	\includegraphics[width = 0.8\linewidth]{images/unisinos.png}
	\legend{Fonte: Adaptado de \citeonline{NBR14724:2011}.}
\end{figure}

\section{Desenho de circuitos}

\begin{circuitikz} \draw
	(0,0)
    %to [battery, v=V1]
    to [R, l=R1, i=$i_{1}$]
    (0,4)
    
    (0,4)
    to [battery, v=V1]
    %to [R, l=R1, i=$i_{1}$]
	(4,4)
    node[anchor=south]{A}
    
    (4,0)
    to [R, l=R2, v>=$V_{R2}$, i=$i_{2}$, *-*]
    (4,4)
    
    (4,0)
    to [short]
    (0,0)
    
    (8,4)
    to [battery, v_>=V2]
    %to [R, l=R3, i>^=$i_{3}$]
	(4,4)
    
    (8,0)
    to [short]
    (4,0)
    node[anchor=north]{G}
    
    (8,0)
    %to [battery, v=V2]
    to [R, l=R3, i=$i_{3}$]
    (8,4)
    ;
\end{circuitikz}


% ||||||||||||||||||||||||||||||||||||||||||||||
% CAPITULO 3
% ||||||||||||||||||||||||||||||||||||||||||||||
\chapter{Materiais, ferramentas e métodos} \label{Metodologia}

\section{Materiais}
	\lipsum[55-57]

\section{Ferramentas}
	\lipsum[90-93]
    
\section{Métodos}
	\lipsum[95-97]
    
% ||||||||||||||||||||||||||||||||||||||||||||||
% CAPITULO 4
% ||||||||||||||||||||||||||||||||||||||||||||||
\chapter{Resultados}

\section{Vestibulum}

\lipsum[21-22]


\section{Pellentesque}

\lipsum[24]

% ||||||||||||||||||||||||||||||||||||||||||||||
% CONCLUSÃO (capítulo sem numeração e presente no sumário)
% ||||||||||||||||||||||||||||||||||||||||||||||
%\chapter*[Conclusão]{Conclusão}
%\addcontentsline{toc}{chapter}{Conclusão}
% Utilize caso a conclusão seja um capitulo sem numeracao.

% ||||||||||||||||||||||||||||||||||||||||||||||
% CAPITULO 5
% ||||||||||||||||||||||||||||||||||||||||||||||
\chapter{Conclusão}

\lipsum[11-12]


\section{Trabalhos futuros}

\lipsum[21-22]

% ----------------------------------------------
% Finaliza a parte no bookmark do PDF para que se inicie o bookmark na raiz e adiciona espaço de parte no Sumário
% ----------------------------------------------
\phantompart

% Arquivo de elementos pós-textuais: referências, apêndices e anexos 
\input{3-postextual}

% ----------------------------------------------
\end{document}
% ----------------------------------------------

% ||||||||||||||||||||||||||||||||||||||||||||||
% ALGUNS EXEMPLOS DE CODIGO LATEX
% ||||||||||||||||||||||||||||||||||||||||||||||

% ----------------------------------------------
%	Figure example
% ----------------------------------------------
% \begin{figure}[htp]
% \centering
% \caption{\label{fig:x} Aqui vai o caption da imagem.}
% \includegraphics[width = 0.9\linewidth ]{images/x.png}
% \legend{Fonte: Adaptado de \citeonline{x}.}
% \end{figure}

% ----------------------------------------------
%	Figure example
% ----------------------------------------------
%\begin{figure}[htp]
%\centering
%\includegraphics[width = 0.5\linewidth ]{dlayer.jpg}
%\caption{\label{fig:1}Complanar waveguide parameters using two dielectric layers \cite{cwccs}.} 
%\end{figure}

% ----------------------------------------------
%	Equation example \ref{eqn:1}
% ----------------------------------------------
%\begin{eqnarray}
%\epsilon _{r} & = & \epsilon _{r}'(1-i\tan(\delta))
%\label{eqn:1}
%\end{eqnarray}

% ----------------------------------------------
%	Complex equations example \ref{eqn:2}
% ----------------------------------------------
%\begin{eqnarray}
%C_{1} & = & 2\epsilon _{0}(\epsilon _{r1}-1)\frac{K(k_{1})}{K(k_{1}')}\nonumber\\
%& = & 2\epsilon _{0}(\epsilon _{r1}'-i\tan(\delta)\epsilon _{r1}'-1)\frac{K(k_{1})}{K(k_{1}')}\nonumber\\
%& = & 2\epsilon _{0}(\epsilon _{r1}'-1)\frac{K(k_{1})}{K(k_{1}')}-i2\epsilon _{0}(\tan(\delta)\epsilon _{r1}')\frac{K(k_{1})}{K(k_{1}')}\\
%C_{2} & = & 2\epsilon _{0}(\epsilon _{r2}-\epsilon _{r1})\frac{K(k_{2})}{K(k_{2}')}\nonumber\\
%& = & 2\epsilon _{0}(\epsilon _{r2}'-i\tan(\delta)\epsilon _{r2}'-\epsilon _{r1}'+i\tan(\delta)\epsilon _{r1}')\frac{K(k_{2})}{K(k_{2}')}\nonumber\\
%& = & 2\epsilon _{0}(\epsilon _{r2}'-\epsilon _{r1}')\frac{K(k_{2})}{K(k_{2}')}-i2\epsilon _{0}(\tan(\delta)\epsilon _{r2}'-\tan(\delta)\epsilon _{r1}')\frac{K(k_{2})}{K(k_{2}')}\\
%C_{vac} & = & 4\epsilon _{0}\frac{K(k_{0})}{K(k_{0}')}
%\label{eqn:2}
%\end{eqnarray}

% ----------------------------------------------
%	Table example \ref{tab:1} 
% ----------------------------------------------
%\begin{table}[htb]
%\caption{Constants and Parameters.}
%\begin{center}
%\begin{tabular}{|c|c|c|c|}
%\hline
%\bfseries CONSTANT & \bfseries VALUE & \bfseries CONSTANT & \bfseries VALUE \\
%\hline \hline
%$\epsilon _{0}$ & 8.8540$\times$10$^{-12}$ & $c$ & 299792458~m/s \\
%\hline
%$\epsilon _{r1}$ & 11.7 & $h_{1}$ & 300~$\mu$m \\
%\hline
%$\epsilon _{r2}$ & 7.5 & $h_{2}$ & 200~nm \\
%\hline
%$\delta _{1}$ & 10$^{-4}$ & $\delta _{2}$ & 10$^{-2}$ $\sim$ 10$^{-3}$ \\
%\hline
%\end{tabular}
%\end{center}
%\label{tab:1}
%\end{table}

% ----------------------------------------------
%	Complex table example \ref{tab:2}
% ----------------------------------------------
%\begin{table}[htb]
%\caption{Results and dimensions.}
%\begin{center}
%\begin{tabular}{|c|c|c|c|c|}
%\hline
%\bfseries $f_{0}$ [MHz] & \bfseries $S$ [$\mu$m] & \bfseries $W$ [$\mu$m] & \bfseries $d$ [cm] & \bfseries $C_{c}$ [fF] \\
%\hline
%\hline
%\multirow{4}{*}{650} & \multirow { 2}{*}{2} & \multirow{ 2}{*}{1} & \multirow{ 2}{*}{9.4966} & 10 \\
%\cline{5-5}
%& & & & 30 \\
%\cline{2-4} \cline{5-5}
%& \multirow { 2}{*}{4.5} & \multirow { 2}{*}{2.3} & \multirow { 2}{*}{9.3155} &         10 \\
%\cline{5-5}
%& & & & 30 \\
%\hline
%\multirow { 4}{*}{6000} &  \multirow { 2}{*}{2} &  \multirow { 2}{*}{1} &  \multirow { 2}{*}{1.0288} &          1 \\
%\cline{5-5}
%& & & & 3 \\
%\cline{2-4} \cline{5-5}
%& \multirow { 2}{*}{4.5} &  \multirow { 2}{*}{2.3} &  \multirow { 2}{*}{1.0092} &          1 \\
%\cline{5-5}
%& & & & 3 \\
%\hline

%\end{tabular}
%\end{center}
%\label{tab:2}
%\end{table}

% ----------------------------------------------
%	Two graphics in one \ref{fig:2}
% ----------------------------------------------
%\begin{figure}[htp]
%\centering
%\begin{tabular}{cc}
%(a) & (b) \\
%\includegraphics[width = 0.5\linewidth ]{DM_6G_4p5_3_6.jpg} &
%\includegraphics[width = 0.5\linewidth ]{DM_650_4p5_10_6.jpg}
%\end{tabular}
%\caption{captions} 
%\label{fig:2}
%\end{figure}

% ----------------------------------------------
%	Four graphics in one table \ref{fig:3}
% ----------------------------------------------
%\begin{figure}[htp]
%\centering
%\begin{tabular}{cc}
%(a) & (b) \\
%\includegraphics[width = 0.5\linewidth ]{DM_650_4p5_10_1.jpg} &
%\includegraphics[width = 0.5\linewidth ]{DM_650_4p5_30_1.jpg} \\
%(c) & (d) \\
%\includegraphics[width = 0.5\linewidth ]{DM_6G_4p5_1_1A.jpg} &
%\includegraphics[width = 0.5\linewidth ]{DM_6G_4p5_3_1.jpg} \\
%\end{tabular}
%\caption{Captions} 
%\label{fig:3}
%\end{figure}

% ----------------------------------------------
%	Quote and footnote
% ----------------------------------------------
%\begin{quote} ``adkfjahsldkfjashflkasdjfhadslkfjhasdlfkjadshflsda
%kjdshflkasjdfhalskfjhadslfksdajhfladskjfhsda
%kajsdfhlaksdjfhasdkl'' \footnote{test footnote}
%\end{quote}

% ----------------------------------------------
%	PDF Annotation
% ----------------------------------------------
%\pdfannot % generic annotation
%width 10cm % the dimension of the annotation can be controlled
%height 0cm % via <rule spec>; if some of dimensions in
%depth 4cm % <rule spec> is not given, the corresponding
% value of the parent box will be used.
%{ %
%/Subtype /Text % text annotation
%/Open true % if given then the text annotation will be opened
%/Contents % text contents
%(write comments in here...)
%}%
%